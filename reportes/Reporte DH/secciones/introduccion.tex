\section{Introducción}
La metodología de Denavit-Hartenberg es una herramienta esencial en la robótica para modelar y analizar la cinemática de robots manipuladores. Gracias a este enfoque, se puede representar matemáticamente la relación entre los eslabones de un robot de manera estandarizada, utilizando cuatro parámetros clave: el ángulo, la distancia, el desplazamiento y la torsión. Estos parámetros permiten definir cómo se conectan y cómo se mueven las distintas partes del robot, desde su base hasta su extremo.

Esta técnica no solo facilita la comprensión de la estructura y el movimiento del robot, sino que también permite resolver problemas fundamentales en la robótica, como la cinemática directa e inversa. La cinemática directa se encarga de determinar la posición y orientación del extremo del robot a partir de los valores de sus articulaciones, mientras que la inversa busca el conjunto de movimientos necesarios para alcanzar una posición específica. Ambos conceptos son fundamentales para el diseño y control de robots.
