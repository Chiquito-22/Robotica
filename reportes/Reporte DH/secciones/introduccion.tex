\section{Introducción}
En la actualidad, la robótica ha evolucionado hasta convertirse en una disciplina clave en múltiples sectores, como la manufactura, la medicina, la exploración espacial y la automatización de procesos. Uno de los aspectos fundamentales que permiten el correcto funcionamiento de los robots es su capacidad para percibir e interpretar el entorno en el que operan. Para lograr esto, los sensores desempeñan un papel esencial, ya que proporcionan la información necesaria para que un robot pueda tomar decisiones, ajustar su comportamiento y realizar tareas con mayor precisión y autonomía.

Los sensores en robótica se pueden clasificar de diversas maneras, dependiendo de la magnitud que miden o del principio de funcionamiento que utilizan. Entre los más comunes se encuentran los sensores de posición, velocidad y aceleración, esenciales para el control del movimiento. También existen sensores de proximidad, que permiten detectar la presencia de objetos sin necesidad de contacto físico, y sensores de fuerza o presión, fundamentales en aplicaciones donde se requiere interacción delicada con el entorno. Además, los sensores ópticos y de visión artificial han cobrado gran relevancia, permitiendo a los robots interpretar imágenes y reconocer patrones, lo que resulta clave en sistemas avanzados de navegación y manipulación de objetos.

El desarrollo y la mejora de estos dispositivos han impulsado avances significativos en la robótica, facilitando la creación de sistemas cada vez más inteligentes y eficientes. Gracias a los sensores, los robots pueden adaptarse a entornos dinámicos, optimizar procesos industriales, mejorar la precisión en intervenciones médicas y desempeñar funciones en lugares de difícil acceso para los humanos.

Lo contenido en tal sección se encuentra en \cite{MagnetometroMundo}