\section{Sensores extras}
\subsection{Giroscopio}
Un giroscopio mide la tasa de rotación alrededor de un eje específico. Se utiliza en muchos dispositivos, como teléfonos inteligentes y consolas de videojuegos, para detectar la orientación y el movimiento. Un tipo común de giroscopio es el giroscopio MEMS (sistema microelectromecánico), que utiliza fuerzas de Coriolis para detectar la rotación. Aplicaciones incluyen:
\begin{itemize}
	\item \textbf{Teléfonos inteligentes}:para la rotación de la pantalla.
	\item \textbf{Drones}:para la estabilización y control.
	\item \textbf{Vehículos}:para sistemas de control de estabilidad.
	
\end{itemize}
\subsection{Acelerómetro}
Un acelerómetro mide la aceleración y la fuerza de gravedad. Los acelerómetros MEMS son comunes en dispositivos electrónicos y utilizan pequeñas estructuras que se mueven cuando hay aceleración. Se pueden encontrar en:
\begin{itemize}
	\item \textbf{Teléfonos inteligentes}:para detectar la orientación y movimiento.
	\item \textbf{Wearables}:para medir la actividad física.
	\item \textbf{Vehículos}: para sistemas de seguridad, como el despliegue de airbags.
	
\end{itemize}
\subsection{Magnetómetro}
Los magnetómetros miden la fuerza y dirección de los campos magnéticos, y son esenciales para las brújulas electrónicas en dispositivos móviles. Existen varios tipos de magnetómetros:
\begin{itemize}
	\item \textbf{Magnetómetros de flujo}:Miden variaciones en el campo magnético.
	\item \textbf{Magnetómetros de efecto Hall}:Utilizan el efecto Hall para medir campos magnéticos. Aplicaciones incluyen:
	\item \textbf{Geofísica}: Para mapear el campo magnético de la Tierra.
	\item \textbf{Sistemas de navegación}:Para detectar la dirección.
	
\end{itemize}
\subsection{LiDAR}
LiDAR utiliza pulsos de láser para medir distancias precisas y crear mapas 3D detallados. Es crucial en el desarrollo de vehículos autónomos, que utilizan LiDAR para detectar obstáculos y comprender el entorno. Otros usos incluyen:
\begin{itemize}
	\item \textbf{Cartografía}:Para crear mapas topográficos detallados.
	\item \textbf{Arqueología}: Para detectar estructuras enterradas.
	\item \textbf{Agricultura}:Para monitorear cultivos y terrenos.
	
\end{itemize}\newpage