\section{Introducción}
La Cinemática Inversa es un pilar fundamental en robótica, ya que permite determinar las configuraciones articulares necesarias para que un robot alcance posiciones específicas en el espacio. Este proceso no sólo es clave para el control del movimiento, sino que también posibilita la identificación y corrección de discrepancias entre la posición teórica deseada y la alcanzada en la práctica. En este reporte, se analiza el desempeño de dos robots simulados en Matlab –el robot 2 y el robot 4–, aplicando algoritmos de Cinemática Inversa para evaluar su comportamiento dinámico.

	\vspace{2em}

El análisis de estas gráficas resulta esencial para comparar la trayectoria planificada con la alcanzada, resaltando el error del objetivo, es decir, la diferencia entre la posición deseada y la obtenida durante la simulación. Este error es un indicador crucial para identificar posibles fuentes de imprecisión, ya sean derivadas del modelado matemático o de la implementación del algoritmo de Cinemática Inversa.