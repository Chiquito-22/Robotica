\subsection{Límites y propiedades dinámicas de las articulaciones} \label{subsec:limites_propiedades}


Para describir la cinemática directa del robot, se utilizaron los parámetros de Denavit-Hartenberg (DH), que permiten definir la posición y orientación relativa entre los eslabones consecutivos del manipulador. Estos parámetros son fundamentales en el análisis y control del robot, ya que simplifican la formulación de la cadena cinemática mediante transformaciones homogéneas.

La Tabla \ref{tab:parametros_robot} muestra los valores utilizados para cada articulación del robot, así como sus respectivas limitaciones mecánicas y dinámicas. Estos parámetros no solo son esenciales para la simulación y control del robot, sino que también permiten implementar algoritmos de planificación de trayectorias, análisis de singularidades, y generación de modelos dinámicos.
