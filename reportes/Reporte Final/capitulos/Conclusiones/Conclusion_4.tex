\section{Oloño Rivera, José Rodrigo}

Al inicio, trabajar con LaTeX fue un reto, ya que no estaba acostumbrado a usar este tipo de herramienta para redactar documentos técnicos. A diferencia de procesadores de texto más comunes, LaTeX requiere entender su estructura, comandos y lógica para poder dar formato correctamente. Sin embargo, con el tiempo y la práctica, fui aprendiendo poco a poco y comencé a valorar su potencia, especialmente para generar documentos bien organizados y con un acabado profesional.

Otro obstáculo que enfrenté fue con mi computadora, ya que por alguna razón los programas que instalo no aparecen visibles de inmediato. Muchas veces tuve que buscar manualmente los accesos directos dentro de las carpetas del sistema para poder ejecutar los programas recién instalados, lo cual hizo el proceso un poco más lento. A pesar de eso, logré adaptarme y sacar adelante el proyecto.
