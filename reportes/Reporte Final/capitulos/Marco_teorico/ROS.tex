\section{ROS} \label{sec:ros}
ROS (Robot Operating System) es un marco de software de código abierto que facilita el desarrollo de aplicaciones robóticas complejas. Aunque su nombre sugiere un sistema operativo tradicional, en realidad ROS actúa como un middleware que ofrece una colección modular de herramientas y bibliotecas para gestionar la comunicación entre procesos (o nodos) distribuidos en red. Proporciona servicios esenciales como la abstracción de hardware, el control de dispositivos de bajo nivel y la administración de paquetes, lo que permite a desarrolladores e investigadores integrar de manera eficiente algoritmos de cinemática, dinámica y control sin tener que reinventar infraestructura básica.

Esta arquitectura modular y flexible, respaldada por una amplia comunidad y soporte multiplataforma, acelera el proceso de simulación y aplicación práctica en robótica. Gracias a herramientas como RViz para visualización y rosbags para el registro de datos, ROS se ha consolidado como la plataforma ideal para probar, validar y desplegar soluciones que integran modelos teóricos y algoritmos de control, facilitando la transición desde la simulación a la implementación real en manipuladores y otros sistemas robóticos.
\begin{figure}[h]
	\centering
	\includegraphics[width=0.5\linewidth]{img/ROS_concepts}
	\caption{Diagrama de comunicación de ROS}
	\label{fig:rosconcepts}
\end{figure}


\subsection{Nodo (Node)}
\subsection{Tema (Topic)}
\subsection{Mensaje (Message)}
\subsection{Servicio (Service)}
\subsection{Gazebo}
\subsection{RViz}
