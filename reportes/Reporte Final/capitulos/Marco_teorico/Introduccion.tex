\chapter{Marco Teórico} 
\label{chap:marco_teorico}

En la robótica, el diseño y control de manipuladores se fundamenta en tres pilares: Cinemática, Dinámica y el uso de ROS. La Cinemática estudia el movimiento, permitiendo calcular la posición y orientación del efector final mediante métodos directos por ejemplo, usando parámetros Denavit-Hartenberg e inversos, esenciales para planificar trayectorias precisas. La Dinámica complementa este análisis al estudiar las fuerzas y torques que producen dichos movimientos, lo que resulta crucial para desarrollar controladores que aseguren un comportamiento estable y seguro frente a cargas y perturbaciones. Finalmente, ROS (Robot Operating System) integra estas disciplinas en un entorno de software modular y flexible, facilitando la simulación y el control real de los manipuladores, y acelerando la transición de modelos teóricos a aplicaciones prácticas.