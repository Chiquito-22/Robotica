\section{Cinemática} \label{sec:cinematica}

La cinemática es la rama de la robótica que se encarga de describir el movimiento de los cuerpos o de los eslabones de un robot sin considerar las causas que lo generan como las fuerzas. Se centra en la representación geométrica del movimiento, permitiendo determinar la posición, orientación, velocidades y aceleraciones de un sistema. En robótica, este estudio se divide en dos grandes enfoques:

Cinemática directa: Calcula la posición y orientación del efector final del robot a partir de un conjunto de variables articulares conocidas.

Cinemática inversa: Realiza el proceso inverso, es decir, a partir de una posición y orientación deseadas se determinan los valores de las variables articulares; sin embargo, este proceso suele presentar dificultades como la existencia de múltiples soluciones o casos en que la configuración deseada no es alcanzable.


\subsection{Cinemática Directa}
La cinemática directa se basa en el análisis geométrico del robot para determinar la posición y orientación del efector final, dado un conjunto de parámetros articulares ángulos o desplazamientos. Uno de los métodos más estandarizados en robótica es el modelo de Denavit-Hartenberg. Este método propone asignar sistemáticamente un sistema de coordenadas a cada eslabón del robot y establecer, a partir de cuatro parámetros característicos:
θ (theta): Ángulo de rotación alrededor del eje z.
d: Desplazamiento a lo largo del eje z.
a: Longitud del eslabón (desplazamiento a lo largo del eje x).
α (alpha): Ángulo de torsión en torno al eje x.

\subsection{Cinemática Diferencial}
La cinemática diferencial se ocupa del estudio de las relaciones entre las velocidades articulares y la velocidad del efector final. Su principal herramienta es el jacobiano, una matriz que se obtiene derivando las ecuaciones de la cinemática directa con respecto a las variables articulares. Matemáticamente, si se tiene una función vectorial   p = f(q) que relaciona la posición y, en ocasiones, la orientación del efector final con el vector de variables articulares q.

\subsection{Cinemática Inversa}
La cinemática inversa se encarga de calcular los valores de las variables articulares que hacen que el robot alcance una posición y orientación específicas. A diferencia de la cinemática directa, el problema inverso suele ser no lineal y puede presentar múltiples soluciones o, en algunas ocasiones, ninguna solución viable. Para abordar esta complejidad se recurre a métodos numéricos que, en general, se basan en técnicas de optimización.

