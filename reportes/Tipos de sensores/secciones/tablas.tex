\section{Sensores externos}
\subsection{Sensores externos tipo de contacto}

\begin{itemize}
	\item \textbf{Interruptores de límite}:Un interruptor de límite (o sensor de límite) es un tipo de sensor de contacto que detecta la presencia o posición de un objeto mediante un actuador mecánico. Funciona como un interruptor eléctrico que se activa cuando un objeto entra en contacto con él, cerrando o abriendo un circuito eléctrico.
	\begin{figure}[h]
		\centering
		\includegraphics[width=6 cm]{img/ilim}
		\caption{Imagen de interruptores de limite}
		\label{fig:ilim}
	\end{figure}
	\item \textbf{Interruptores neumáticos}:Los interruptores neumáticos son dispositivos que utilizan aire comprimido para activar o desactivar un circuito, en lugar de depender de contactos eléctricos o mecánicos. Funcionan detectando cambios en la presión del aire y pueden actuar como sensores o interruptores de control en sistemas industriales.
	\begin{figure}[h]
		\centering
		\includegraphics[width=6 cm]{img/inem}
		\caption{Imagen de un interruptor neumatico}
		\label{fig:inem}
	\end{figure}
	\item \textbf{Sensores Piezoeléctricos}:Los sensores piezoeléctricos son dispositivos que utilizan materiales piezoeléctricos (como el cuarzo, titanato de plomo o cerámicas especiales) para convertir fuerzas mecánicas, presión o vibraciones en señales eléctricas. Esto ocurre debido al efecto piezoeléctrico (es la capacidad de ciertos materiales (como el cuarzo o cerámicas especiales) de generar una carga eléctrica cuando se deforman mecánicamente), donde ciertos materiales generan una carga eléctrica cuando se someten a deformaciones mecánicas.\newpage
	\begin{figure}[h]
		\centering
		\includegraphics[width=6 cm]{img/spiez}
		\caption{Imagen desensores piezoeléxtricos}
		\label{fig:spiez}
	\end{figure}
	\item \textbf{Transductores de presión}:Los transductores de presión son dispositivos que convierten la presión de un fluido (líquido o gas) en una señal eléctrica, mecánica o neumática. Se utilizan para medir la presión en sistemas industriales, automotrices, médicos y más
	\begin{figure}[h]
		\centering
		\includegraphics[width=5 cm]{img/tpres}
		\caption{Imagen de un transductor de presión}
		\label{fig:tpres}
	\end{figure}
\end{itemize}

\subsection{Sensores externos tipo sin contacto}
\begin{itemize}
	\item \textbf{Sensores de Proximidad}:Los sensores de proximidad son dispositivos que detectan la presencia o ausencia de un objeto sin necesidad de contacto físico. Funcionan mediante distintos principios físicos, como electromagnetismo, ultrasonido o luz.
	\begin{figure}[h]
		\centering
		\includegraphics[width=10 cm]{img/Sprox}
		\caption{Imagen de distintos sensores de proximidad}
		\label{fig:sprox}
	\end{figure}\newpage
	\item \textbf{Sensores de microondas}:Los sensores de microondas son dispositivos que emiten ondas electromagnéticas de alta frecuencia y analizan su reflejo para detectar movimiento, proximidad o cambios en el entorno. Funcionan con el efecto Doppler, midiendo variaciones en la frecuencia de las ondas reflejadas.
	\begin{figure}[h]
		\centering
		\includegraphics[width=6 cm]{img/smic}
		\caption{Funcionamiento de un sensor de microondas}
		\label{fig:smic}
	\end{figure}
	\item \textbf{Sensores ultrasónicos }: Los sensores ultrasónicos son dispositivos que emiten ondas de sonido de alta frecuencia (ultrasonido) y miden el tiempo que tarda el eco en regresar después de reflejarse en un objeto. Con este dato, pueden calcular distancias o detectar la presencia de objetos sin contacto físico
	\begin{figure}[h]
		\centering
		\includegraphics[width=13 cm]{img/sultra}
		\caption{Distintos tipos de sensores ultrasonicos}
		\label{fig:sultra}
	\end{figure}
	\item \textbf{Sensores láser }: Los sensores láser son dispositivos que emiten un haz de luz láser para detectar objetos, medir distancias o analizar superficies con alta precisión. Funcionan mediante el principio de reflexión, el efecto Doppler o la interferometría.
	\begin{figure}[h]
		\centering
		\includegraphics[width=8 cm]{img/slas}
		\caption{Sensor laser}
		\label{fig:slas}
	\end{figure}\newpage
	\item \textbf{Sensores de visión}:Los sensores de visión son dispositivos que capturan imágenes y procesan información visual para detectar, inspeccionar o identificar objetos en tiempo real. Usan cámaras, algoritmos de procesamiento de imagen e inteligencia artificial para analizar el entorno.
	\begin{figure}[h]
		\centering
		\includegraphics[width=8 cm]{img/svis}
		\caption{Imagen de sensores de visión}
		\label{fig:svis}
	\end{figure}\newpage
\end{itemize}